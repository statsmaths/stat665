\documentclass[xetex,mathserif,serif,aspectratio=169]{beamer}

\input{../import.tex}
\usepackage[]{algorithm2e}
\usepackage{../kbordermatrix}

\begin{document}

%%%%%%%%%%%%%%%%%%%%%%%%%%%%%%%%%%%%%%%%%%%%%%%%%%%
\begin{frame}[fragile] \frametitle{} \oldB \small

\vfill

{\fontsize{0.7cm}{0cm}\selectfont Lecture 17 \\\vspace{0.2cm}
Computer Vision with CNNs}\\\vspace{0.5cm}
04 April 2016

\vspace{2cm}

\begin{minipage}{0.6\textwidth}
Taylor B. Arnold \\
Yale Statistics \\
STAT 365/665
\end{minipage}
\hfill
\begin{minipage}{0.3\textwidth}\raggedleft
\includegraphics[scale=0.3]{../yale-logo.png}
\end{minipage}%

\end{frame}

%%%%%%%%%%%%%%%%%%%%%%%%%%%%%%%%%%%%%%%%%%%%%%%%%%%
\begin{frame}[fragile] \frametitle{} \oldB \small

Notes:
\begin{itemize}
\item Problem set 6 is online and due \textbf{this} Friday, April 8th
\item Problem set 7 is online and due \textbf{next} Friday, April 15th
\item No class next week
\end{itemize}

\end{frame}

%%%%%%%%%%%%%%%%%%%%%%%%%%%%%%%%%%%%%%%%%%%%%%%%%%%
\begin{frame}[fragile] \frametitle{} \oldB \small

\yblue{\textbf{Convolutional Models in Computer Vision}}

There is a long history of specific advances and uses of convolutional
neural networks. Today, I'll focus on the following set of models:
\begin{itemize}
\item LeNet-5 (1998)
\item AlexNet (2012)
\item OverFeat (2013)
\item GoogLeNet (2014)
\item VGG-16, VGG-19 (2014)
\item ResNet-50, ResNet-101, ResNet-152 (2015)
\item SqueezeNet (2016)
\item ResNet-200, ResNet-1001 (2016)
\end{itemize}
When you hear about these models people are sometimes referring to the exact
architecture and weights, sometimes to architecture without the weights, and
sometimes just to general approach.

\end{frame}

%%%%%%%%%%%%%%%%%%%%%%%%%%%%%%%%%%%%%%%%%%%%%%%%%%%
\begin{frame}[fragile] \frametitle{} \oldB \small

\yblue{\textbf{LeNet-5 (1998)}}

LeNet was one of first models to really show the power of convolutional
neural networks. It was first applied to the MNIST-10 dataset, created
by a similar group of individuals:
\begin{quote}
LeCun, Y., Bottou, L., Bengio, Y. and Haffner, P., 1998. Gradient-based
learning applied to document recognition. Proceedings of the IEEE, 86(11),
pp.2278-2324.
\end{quote}


\end{frame}

%%%%%%%%%%%%%%%%%%%%%%%%%%%%%%%%%%%%%%%%%%%%%%%%%%%
\begin{frame}[fragile] \frametitle{} \oldB \small

\yblue{\textbf{AlexNet (2012)}}

University of Toronto
\begin{quote}
Krizhevsky, Alex, Ilya Sutskever, and Geoffrey E. Hinton. "Imagenet classification with deep convolutional neural networks." In Advances in neural information processing systems, pp. 1097-1105. 2012.
\end{quote}

\end{frame}




%%%%%%%%%%%%%%%%%%%%%%%%%%%%%%%%%%%%%%%%%%%%%%%%%%%
\begin{frame}[fragile] \frametitle{} \oldB \small

\yblue{\textbf{Visualizing CNNs (2013)}}


\begin{quote}
Zeiler, Matthew D., and Rob Fergus. "Visualizing and understanding convolutional networks." Computer vision–ECCV 2014. Springer International Publishing, 2014. 818-833.
\end{quote}


\end{frame}

%%%%%%%%%%%%%%%%%%%%%%%%%%%%%%%%%%%%%%%%%%%%%%%%%%%
\begin{frame}[fragile] \frametitle{} \oldB \small

\yblue{\textbf{OverFeat (2013)}}

NYU
\begin{quote}
Sermanet, Pierre, David Eigen, Xiang Zhang, Michaël Mathieu, Rob Fergus, and Yann LeCun. "Overfeat: Integrated recognition, localization and detection using convolutional networks." arXiv preprint arXiv:1312.6229 (2013).
\end{quote}


\end{frame}

%%%%%%%%%%%%%%%%%%%%%%%%%%%%%%%%%%%%%%%%%%%%%%%%%%%
\begin{frame}[fragile] \frametitle{} \oldB \small

\yblue{\textbf{GoogLeNet (2014)}}

Google
\begin{quote}
Szegedy, Christian, Wei Liu, Yangqing Jia, Pierre Sermanet, Scott Reed, Dragomir Anguelov, Dumitru Erhan, Vincent Vanhoucke, and Andrew Rabinovich. "Going deeper with convolutions." In Proceedings of the IEEE Conference on Computer Vision and Pattern Recognition, pp. 1-9. 2015.
\end{quote}


\end{frame}

%%%%%%%%%%%%%%%%%%%%%%%%%%%%%%%%%%%%%%%%%%%%%%%%%%%
\begin{frame}[fragile] \frametitle{} \oldB \small

\yblue{\textbf{VGG-16, VGG-19 (2014)}}

Oxford
\begin{quote}
Simonyan, Karen, and Andrew Zisserman. "Very deep convolutional networks for large-scale image recognition." arXiv preprint arXiv:1409.1556 (2014).
\end{quote}


\end{frame}

%%%%%%%%%%%%%%%%%%%%%%%%%%%%%%%%%%%%%%%%%%%%%%%%%%%
\begin{frame}[fragile] \frametitle{} \oldB \small

\yblue{\textbf{ResNet-50, -101, -152 (2015)}}

Microsoft
\begin{quote}
He, Kaiming, Xiangyu Zhang, Shaoqing Ren, and Jian Sun. "Deep Residual Learning for Image Recognition." arXiv preprint arXiv:1512.03385 (2015).
\end{quote}


\end{frame}

%%%%%%%%%%%%%%%%%%%%%%%%%%%%%%%%%%%%%%%%%%%%%%%%%%%
\begin{frame}[fragile] \frametitle{} \oldB \small

\yblue{\textbf{ResNet-200, -1001 (2016)}}

Microsoft's update to last year's model. Posted only two
weeks ago!
\begin{quote}
He, Kaiming, et al. "Identity Mappings in Deep Residual Networks." arXiv preprint arXiv:1603.05027 (2016).
\end{quote}

\end{frame}

%%%%%%%%%%%%%%%%%%%%%%%%%%%%%%%%%%%%%%%%%%%%%%%%%%%
\begin{frame}[fragile] \frametitle{} \oldB \small

\yblue{\textbf{SqueezeNet (2016)}}

\begin{quote}
Iandola, Forrest N., et al. "SqueezeNet: AlexNet-level accuracy with
50x fewer parameters and< 1MB model size." arXiv preprint arXiv:1602.07360 (2016).
\end{quote}


\end{frame}


\end{document}







